\chapter*{Abstract}

The thesis provides insight into the confluence of financial mathematical models and advanced machine-learning techniques. It seeks to explore the areas of pricing, hedging, and volatility modelling by transcending the limitations of traditional models and leveraging the cutting-edge advancements of deep learning in handling complex financial calculations.

Grounded on the conception of neural stochastic differential equations (SDEs), the research demonstrates a novel approach to robust pricing and hedging technique that provides dependable bounds, ensuring enhanced market realism and improved risk profiles. A crucial advantage is the flexible calibration under risk-neutral and real-world measures, with the capability to augment the accuracy of simulating diverse market scenarios.

The discourse continues with a deep learning-based numerical algorithm that presents an adept resolution to some of the challenges associated with numerical solutions to path-dependent partial differential equations arising from Rough volatility. The method hinges on neural networks of a reservoir type and displays compelling theoretical convergence properties while offering practical regression-based solutions.

The study further delves into the field of Monte-Carlo estimators, more precisely of variance reduction in stochastic volatility models. Focusing on the Heston model, it unfolds a comprehensive analysis of importance sampling, based on large and moderate deviation principles. The findings are illustrated through theoretical proposals and empirical trials.

Lastly, the thesis tries to address a considerable gap in the equity markets' rough volatility model, the Rough Bergomi. It presents a generalised model that diverges from the log-normality constraint, incorporating self-similarity and stationary increments, hopefully offering a refined reflection of VIX market data.

In summary, we hope this research casts a new light on the symbiosis between financial mathematics and modern data science techniques. Its goal is to propel a paradigm shift in our understanding of financial markets modelling and risk management, magnifying the transformative potential of deep learning techniques on traditional mathematical models leveraged in the financial industry.