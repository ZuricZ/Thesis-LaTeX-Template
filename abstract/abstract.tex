\abstracttitleandname
\vspace*{-4\baselineskip}
\chapter*{Abstract}
\vspace*{-3\baselineskip}
\begin{spacing}{1.116}
{\small
This thesis explores the confluence of financial mathematical models and advanced machine\nobreakdash-learning techniques, specifically focusing on pricing, hedging and volatility modelling. By transcending the limitations of traditional models, it leverages cutting-edge advancements in deep learning to handle complex financial calculations, providing valuable insights in these areas.

One significant contribution is the development of neural stochastic differential equations, a novel approach that enhances pricing and hedging by capturing realistic market behaviour while providing reliable bounds. Notably, the approach offers flexible calibration under risk-neutral and real-world measures, enabling more accurate simulations of diverse market scenarios.

Next, to tackle the challenges posed by path-dependent partial differential equations arising from rough volatility models, this thesis introduces a numerical algorithm utilizing reservoir-type neural networks. The algorithm merges theoretical convergence properties with practical regression-based solutions, effectively overcoming some of the limitations of traditional numerical methods.

Furthermore, this study delves into variance reduction in stochastic volatility models, with a particular focus on the Heston model. It presents a comprehensive analysis of importance sampling, based on large and moderate deviation principles. The variance reduction of the resulting Monte Carlo estimators is demonstrated through theoretical guarantees and empirical studies.

Lastly, the thesis aims to address a significant gap in a rough volatility model used in equity markets, namely the Rough Bergomi model. It proposes a generalized model that departs from the log-normality constraint, incorporating self-similarity and stationary increments. Through this refinement, it seeks to provide a more accurate reflection of VIX market data.

In summary, we hope this research casts a new light on the symbiosis between financial mathematics and modern data science techniques. Its goal is to propel a paradigm shift in our understanding of modelling financial markets, magnifying the transformative potential of deep learning techniques on traditional mathematical models currently used in the financial industry.
}
\enlargethispage{2\baselineskip}
\end{spacing}



% ----------

% Overall, this research aims to cast a new light on the symbiosis between financial mathematics and modern data science techniques. By propelling a paradigm shift in modelling financial markets, it magnifies the transformative potential of deep learning techniques on traditional mathematical models currently employed in the financial industry.

% This thesis provides insight into the confluence of financial mathematical models and advanced machine-learning techniques. It seeks to explore the areas of pricing, hedging and volatility modelling by transcending the limitations of traditional models and leveraging the cutting-edge advancements of deep learning in handling complex financial calculations.

% A key contribution is the development of neural stochastic differential equations, a novel approach that enhances pricing and hedging by providing reliable bounds and capturing realistic market behaviour. Notably, these equations offer flexible calibration under both risk-neutral and real-world measures, enabling more accurate simulations of diverse market scenarios.

% To address the challenges of path-dependent partial differential equations arising from rough volatility models, this thesis introduces a machine-learning-based numerical algorithm. The algorithm, based on reservoir-type neural networks, combines theoretical convergence properties with practical regression-based solutions, overcoming the limitations of traditional numerical methods.

% Next, our study further delves into variance reduction in stochastic volatility models. Focusing on the Heston model, it unfolds a comprehensive analysis of importance sampling, based on large and moderate deviation principles. The research offers theoretical guarantees and empirical insights into importance sampling approaches for minimizing variance in Monte Carlo estimators.

% Lastly, the thesis tries to address a considerable gap in the equity markets' rough volatility model, the Rough Bergomi. It presents a generalised model that diverges from the log-normality constraint, incorporating self-similarity and stationary increments, hopefully offering a refined reflection of VIX market data.

% In summary, we hope this research casts a new light on the symbiosis between financial mathematics and modern data science techniques. Its goal is to propel a paradigm shift in our understanding of modelling financial markets, magnifying the transformative potential of deep learning techniques on traditional mathematical models currently used in the financial industry.

% ----------


% This thesis provides insight into the confluence of financial mathematical models and advanced machine-learning techniques. It seeks to explore the areas of pricing, hedging, and volatility modelling by transcending the limitations of traditional models and leveraging the cutting-edge advancements of deep learning in handling complex financial calculations.

% Grounded on the conception of neural stochastic differential equations, the research demonstrates a novel approach to robust pricing and hedging that provides dependable bounds, while ensuring enhanced market realism. A crucial advantage, compared to classical techniques, is a flexible calibration procedure under both risk-neutral and real-world measures, with the capability to augment the accuracy of simulating diverse market scenarios.

% The discourse continues with a machine-learning-based numerical algorithm that presents an adept resolution to some of the challenges associated with numerical solutions to path-dependent partial differential equations that arise from rough volatility models. The method hinges on neural networks of a reservoir type and displays compelling theoretical convergence properties while offering practical regression-based solutions.

% The study further delves into the field of Monte-Carlo estimators, more precisely of variance reduction in stochastic volatility models. Focusing on the Heston model, it unfolds a comprehensive analysis of importance sampling, based on large and moderate deviation principles. The findings are illustrated through theoretical guarantees and empirical trials.

% Lastly, the thesis tries to address a considerable gap in the equity markets' rough volatility model, the Rough Bergomi. It presents a generalised model that diverges from the log-normality constraint, incorporating self-similarity and stationary increments, hopefully offering a refined reflection of VIX market data.

% In summary, we hope this research casts a new light on the symbiosis between financial mathematics and modern data science techniques. Its goal is to propel a paradigm shift in our understanding of financial markets modelling and risk management, magnifying the transformative potential of deep learning techniques on traditional mathematical models currently used in the financial industry.