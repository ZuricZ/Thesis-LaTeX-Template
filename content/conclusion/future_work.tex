In this section, we discuss potential areas of research and development that can build upon the findings and insights presented in each chapter. For Chapter~\ref{chap:reschap1}, future work may involve exploring the application of neural SDEs in other financial derivative pricing and hedging contexts. Towards the end of Section~\ref{sec nsdes}, we demonstrated a connection with generative modelling and optimal transport, which opens an entire avenue to new ideas. Another possible segue from this topic would be to instead consider neural SPDEs, which could then be used to model the dynamics of the implied volatility surface through time.


In Chapter~\ref{chap:reschap2}, further research could, naturally, focus on extending the random neural network approach to pricing BSPDEs for rough volatility to a higher-dimensional setting. Additionally, incorporating market data analysis and performance comparison with other existing pricing methods could provide valuable insights. Continuing the theme of generative modelling of Chapter~\ref{chap:reschap2}, we mention Generative
Adversarial Networks~\cite{Goodfellow2020GenerativeNetworks} (or GANs for short). GANs are notoriously hard to train because of their inherent instability between the two competing
agents. There have been many attempts trying to reconcile their weaknesses~\cite{Arjovsky2017WassersteinNetworks, Li2017MMDNetwork}, but many of them lack rigour and theoretical guarantees. Recently, RWNNs have made their comeback after their, also because of their tractability and robustness to noise. Considering all this, an RWNN-GAN would be an interesting combination, which could give an insight into the inner workings of GANs and \textit{perhaps} deliver a closed-form solution to the precarious min-max problem.


In our work, in Chapter~\ref{chap:reschap3}, we focused on variance reduction using importance sampling via large and moderate
deviations for certain types of stochastic volatility models. An obvious first step is of course a generalisation to a more general class of models, perhaps including rough volatility models. The generalisation to models with paths rougher than those of Brownian motion is not straightforward and comes with its own caveat. As an example see~\cite{Jacquier2018PathwiseModel, Jacquier2022LargeSystems}, where authors devise a large deviation principle using small-noise approximation and~\cite{Jacquier2022LargeSystems} which provides a unified treatment of pathwise large and moderate deviation principles for a general class of multidimensional stochastic Volterra equations with singular kernels. 


As for Chapter~\ref{chap:reschap4}, ongoing work can include refining and expanding the grey Bergomi model to capture additional stylized facts of volatility and conducting empirical studies to validate its performance against market data. Replacing a standard fractional Brownian motion in the definition of grey Brownian motion with the one of Riemann-Liouville type is of particularly high interest from a practical standpoint. Overall, future work aims to advance the existing research by addressing limitations and applying the proposed model to real-world financial scenarios.