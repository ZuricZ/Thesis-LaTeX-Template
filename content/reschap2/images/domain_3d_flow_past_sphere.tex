\documentclass[11pt,landscape]{article}

% Set page legths
\special{papersize=50cm,50cm}
\hoffset-0.8in
\voffset-0.8in
\setlength{\paperwidth}{50cm}
\setlength{\paperheight}{50cm}
\setlength{\textwidth}{45cm}
\setlength{\textheight}{45cm}
\topskip0cm
\setlength{\headheight}{0cm}
\setlength{\headsep}{0cm}
\setlength{\topmargin}{0cm}
\setlength{\oddsidemargin}{0cm}
% set the pagestyle to empty (removing pagenumber etc)
\pagestyle{empty}

% load packages:
\usepackage{amsmath}
\usepackage{amssymb}
\usepackage{amstext}
\usepackage{amsfonts}
\usepackage{units}

% TikZ:
\usepackage{tikz}
\usetikzlibrary{shapes.geometric, arrows, intersections, through}
\usetikzlibrary{decorations.text, decorations.shapes, backgrounds}
\usetikzlibrary{positioning}
\usetikzlibrary{calc,fadings,decorations.pathreplacing}
\usetikzlibrary{calc}
\usepackage{pgfplots}
%\usepgfplotslibrary{external}
%\tikzexternalize

% Use newest spacing options (from v. 1.3 on)
\pgfplotsset{compat=newest}
\pgfplotsset{width=10cm}
\pgfplotsset{height=6cm}

% Remove white space from generated pdf,
% thus otaining a pdf with only the picture that can
% easily be included in a(nother) tex-document via the usual \includegraphic command.
% Benefit: 1. you can keep your pictures organised in a subfolder and
%          2. the picture remains a vector graphic :)
\usepackage[active, tightpage]{preview}
\PreviewEnvironment{tikzpicture}
\setlength\PreviewBorder{3pt}
% Alternatively, delete the three lines above and run 'pdfcrop filename.pdf',
% the result should be the same.

% defining position:
\newdimen\nodeSize
\nodeSize=0mm
\newdimen\nodeDist
\nodeDist=20pt

\begin{document}
  \centering
  \begin{tikzpicture}[scale=4,
      node/.style={draw,circle,inner sep=0,outer sep=0,minimum size=\nodeSize,node distance=\nodeDist},
      position/.style args={#1:#2 from #3}{
              at=(#3.#1), anchor=#1+180, shift=(#1:#2)
      },
      line/.style={-,thick},
      helpmeasure/.style={line,thin,gray},
      backline/.style={line,very thick,loosely dashed,color=black,opacity=0.75},
      frontline/.style={line,very thick,color=black},
      backface/.style={color=blue!20,opacity=0.8},
      frontface/.style={color=blue!20,opacity=0.5},
      corner/.style={circle,inner sep=0pt, outer sep=0pt, minimum size=0pt},
      measure/.style={<->,>=stealth',shorten >=1pt,gray!70!black,densely dashed},
      extended line/.style={shorten >=-#1,shorten <=-#1},
      extended line/.default=1cm,
      axisarrow/.style={->,>=stealth',shorten >=1pt,thin,gray}
      ]

    % defining colors:
    \colorlet{measurecolor}{gray!70!black}
    \colorlet{spherecolor}{red!75!white}
    \colorlet{wallcolor}{blue!20}
    \colorlet{hiddencolor}{gray!80!black}

	  %% Vanishing points for perspective handling
	  \coordinate (P1) at (5cm,4cm); % left vanishing point (To pick)
	  \coordinate (P2) at (-12.5cm,4cm); % right vanishing point (To pick)

	  %% (A1) and (A2) defines the 2 central points of the cuboid
	  \coordinate (A1) at (0em,0cm); % central top point (To pick)
	  \coordinate (A2) at (0em,-2cm); % central bottom point (To pick)
    
	  %% (A3) to (A8) are computed given a unique parameter (or 2) .8
	  % You can vary .8 from 0 to 1 to change perspective on left side
	  \coordinate (A3) at ($(P1)!.8!(A2)$); % To pick for perspective 
	  \coordinate (A4) at ($(P1)!.8!(A1)$);

	  % You can vary .8 from 0 to 1 to change perspective on right side
	  \coordinate (A7) at ($(P2)!.7!(A2)$);
	  \coordinate (A8) at ($(P2)!.7!(A1)$);

	  %% Automatically compute the last 2 points with intersections
	  \coordinate (A5) at
	    (intersection cs: first line={(A8) -- (P1)},
			  second line={(A4) -- (P2)});
	  \coordinate (A6) at
	    (intersection cs: first line={(A7) -- (P1)}, 
			  second line={(A3) -- (P2)});

    %% Possibly draw back faces
    \fill[backface,color=blue!75] (A2) -- (A3) -- (A6) -- (A7) -- cycle; % face 6
    \fill[backface,color=blue!50] (A3) -- (A4) -- (A5) -- (A6) -- cycle; % face 3
    \fill[backface,color=blue!30] (A5) -- (A6) --  (A7) -- (A8) -- cycle; % face 4
	  % back (hidden) lines:
	  \draw[backline] (A5) -- (A6);
	  \draw[backline] (A3) -- (A6);
	  \draw[backline] (A7) -- (A6);

    % draw sphere before drawing front faces and lines:
    \coordinate (SC) at (-2,0.45); % centre point of sphere
    \def\R{0.25} % sphere radius
    \fill[ball color=spherecolor, opacity=0.75] (SC) circle (\R); % 3D lighting effect
    \draw[backline,solid] (SC) circle (\R);
    % also drawing helpmeasure lines before drawing the rest of the (fluid) box:
    \node[node,position=90:{1.15} from SC] (SCm1) {};
    \node[node,position=90:{5.5} from SC] (SCm2) {};
    \path (A5) -- node[node,midway] (A5A8c) {} (A8);
    \path (A6) -- node[node,midway] (A6A7c) {} (A7);
    \path (A6A7c) -- node[node,midway] (inletC) {} (A5A8c);
    \node[node,position=90:{5.5} from inletC] (SCm3a) {};
    % now construct 2 paths, one up from inletC, and one to the left from above the sphere
    % then create a node at the intersection of those paths which we'll use for the measuring line:
    \path[name path=inletC--SCm3] (inletC) -- (SCm3a);
    \node[node,position=180:{6} from SCm2] (SCm3b) {};
    \path[name path=SCm2--SCm3b] (SCm2) -- (SCm3b);
    \path[name intersections={of=inletC--SCm3 and SCm2--SCm3b,by=SCm3}];
    \draw[helpmeasure,opacity=0.75] (SCm1) -- (SCm2);
    \node[node,position=90:{1.1} from SCm3] (SCm3c) {};
    \draw[helpmeasure] (inletC) -- (SCm3c);
    \draw[measure,postaction={decoration={text along path, text color=measurecolor, text={|\Large|1m}, text align={align=center}, raise=3pt}, decorate}] (SCm3c) -- (SCm2);
    % same procedure for the measurement line: sidewall to sphere
    % first, node below sphere, then helpmeasure line from below sphere to bottom surface:
    \node[node,position=270:{1.15} from SC] (SCmb1) {};
    \node[node,position=270:{3} from SCmb1] (SCmb2) {};
    % for that to work, we need to find the intersection between the downwards vector from the sphere
    % and the vector going from the centre point of the bottom inletface line, and the bottom outlet face line:
    \path (A2) -- node[node,midway] (A2A3c) {} (A3);
    \path[name path=inlet--outlet--bottom] (A6A7c) -- (A2A3c);
    \path[name path=SCmb1--SCmb2] (SCmb1) -- (SCmb2);
    \path[name intersections={of=SCmb1--SCmb2 and inlet--outlet--bottom,by=below-sphere}];
    % finally we can draw the line from right below the sphere to the bottom face of the box:
    \draw[helpmeasure] (SCmb1) -- (below-sphere);
    % now drawing a path somewhat parallel to the inlet (bottom) line in z-direction, through node 'below-sphere':
    \node[node,position=211:{4} from below-sphere] (bottom-SCm1) {};
    \path[extended line=4cm,name path=below-sphere-z-path] (bottom-SCm1) -- (below-sphere);
    % for testing:
    %\draw[very thick, red, extended line=4cm] (bottom-SCm1) -- (below-sphere);
    % and the two paths for the bottom lines going from inlet to outlet surface:
    \path[name path=A6A3] (A6) -- (A3);
    \path[name path=A7A2] (A7) -- (A2);
    % find intersections:
    \path[name intersections={of=below-sphere-z-path and A7A2, by=SCm-on-A7A2}];    
    \node[node,position=31:{4} from below-sphere] (bottom-SCm2) {};
    \path[name path=below-sphere-z-path-2] (below-sphere) -- (bottom-SCm2);
    \path[name intersections={of=below-sphere-z-path-2 and A6A3, by=SCm-on-A6A3}];
    % now all nodes are defined, so we can draw the measurement lines:
    \draw[measure,postaction={decoration={text along path, text color=black, text={|\Large|1m}, text align={align=center}, raise=3pt}, decorate}] (SCm-on-A7A2) -- (below-sphere);
    \draw[measure,postaction={decoration={text along path, text color=black, text={|\Large|1m}, text align={align=center}, raise=3pt}, decorate}] (below-sphere) -- (SCm-on-A6A3);

    % front faces:
    \fill[frontface,color=blue!60,opacity=0.5] (A1) -- (A8) -- (A7) -- (A2) -- cycle; % face 1
    \fill[frontface,color=blue!45,opacity=0.5] (A1) -- (A2) -- (A3) -- (A4) -- cycle; % f2
    \fill[frontface,color=blue!75,opacity=0.5] (A1) -- (A4) -- (A5) -- (A8) -- cycle; % f5

	  %% Possibly draw front lines
	  \draw[frontline] (A1) -- (A2);
	  \draw[frontline] (A3) -- (A4);
	  \draw[frontline] (A7) -- (A8);
	  \draw[frontline] (A1) -- (A4);
	  \draw[frontline] (A1) -- (A8);
	  \draw[frontline] (A2) -- (A3);
	  \draw[frontline] (A2) -- (A7);
	  \draw[frontline] (A4) -- (A5);
	  \draw[frontline] (A8) -- (A5);
	  
    % measurements:
    % measure bottom line in x direction:
    \node[node,position=210:{\nodeDist} from A7] (A7xm) {};
    \node[node,position=230:{\nodeDist} from A2] (A2xm) {};
    \draw[helpmeasure] (A2xm) -- (A2);
    \draw[helpmeasure] (A7xm) -- (A7);
    \draw[measure,postaction={decoration={text along path, text color=measurecolor, text={|\Large|3m}, text align={align=center}, raise=3pt}, decorate}] (A7xm) -- (A2xm);
    % measure right bottom in z direction:
    \node[node,position=-26:{\nodeDist} from A2] (A2zm) {};
    \node[node,position=-22:{\nodeDist} from A3] (A3zm) {};
    \draw[helpmeasure] (A2zm) -- (A2);
    \draw[helpmeasure] (A3zm) -- (A3);
    \draw[measure,postaction={decoration={text along path, text color=measurecolor, text={|\Large|2m}, text align={align=center}, raise=3pt}, decorate}] (A2zm) -- (A3zm);
    % measure right side in y direction:
    \node[node,position=0:{\nodeDist} from A3] (A3ym) {};
    \node[node,position=0:{\nodeDist} from A4] (A4ym) {};
    \draw[helpmeasure] (A3ym) -- (A3);
    \draw[helpmeasure] (A4ym) -- (A4);
    \draw[measure,postaction={decoration={text along path, text color=measurecolor, text={|\Large|2m}, text align={align=center}, raise=3pt}, decorate}] (A3ym) -- (A4ym);

    % adding axis:
    \node[corner,at=(A7), shift=(-75:4cm)] (axiso) {};
    \draw[axisarrow] (axiso)  -- +(330:0.35cm) node[corner,label=above:$x$] (axisox) {};
    \draw[axisarrow] (axiso)  -- +(35:0.35cm) node[corner,label=above:$y$] (axisoy) {};
    \draw[axisarrow] (axiso)  -- +(90:0.35cm) node[corner,label=above:$z$] (axisoz) {};

  \end{tikzpicture}
\end{document}
